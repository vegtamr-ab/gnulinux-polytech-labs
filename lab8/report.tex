\documentclass[a4paper,11pt]{article}

\usepackage{wrapfig}
\usepackage{amsmath}
\usepackage{listings}
\usepackage{color}
\usepackage[dvipsnames]{xcolor}
%Russian-specific packages
%--------------------------------------
\usepackage[T2A]{fontenc}
\usepackage[utf8]{inputenc}
\usepackage[russian, english]{babel}
%--------------------------------------

\title{Отчёт по лабораторной работе №8 по дисциплине GNU/Linux}
\author{Андрей Бареков}
\date{\today}

\lstdefinestyle{mycode}{
    backgroundcolor=\color{White},   
    commentstyle=\color{ForestGreen},
    keywordstyle=\color{Blue},
    numberstyle=\tiny\color{Maroon},
    stringstyle=\color{Plum},
    basicstyle=\ttfamily\footnotesize,
    breakatwhitespace=false,         
    breaklines=true,                 
    captionpos=b,                    
    keepspaces=true,                 
    numbers=left,                    
    numbersep=5pt,                  
    showspaces=false,                
    showstringspaces=false,
    showtabs=false,                  
    frame=single,
    tabsize=2,
}

\begin{document}
\maketitle
\newpage
\lstset{style=mycode}

\section{Задание}
  Драйвер поддерживает чтение и запись сообщений в него через существующие утилиты POSIX:
    \begin{lstlisting}
echo "message text_message" > /dev/chardev
    \end{lstlisting}
  Хранит только одно сообщение. Поддерживает функцию удаления сообщения: \textit{msg\_delete}.

\section{Цель работы}
  Драйвер, соответствующий требованиям, указанным в задании.

\section{Задачи}
  \begin{enumerate}
    \item Написать код драйвера.
    \item Собрать драйвер.
    \item Подключить драйвер.
    \item Проверить правильность работы драйвера.
  \end{enumerate}

\section{Платформы}
  \subsection{Аппаратная платформа}
    CPU: Intel i5-8250U (8) @ 3.400GHz

  \subsection{Программная платформа}
    OS: Linux Manjaro x86\_64

\newpage
\section{Ход работы}
  \subsection{Реализация драйвера. Изменения.}
    Для удобства код драйвера был разделён на модули:
    \begin{itemize}
      \item \textbf{chardev.h} - основной модуль, в котором содержится документация драйвера
      \item \textbf{initialize.c} - модуль с инициализацией и деинициализацией драйвера
      \item \textbf{open.c} - модуль с открытием драйвера
      \item \textbf{release.c} - модуль с закрытием драйвера
      \item \textbf{read.c} - модуль с чтением
      \item \textbf{write.c} - модуль с записью \\
    \end{itemize}
    
    Изменено не очень приятное оформление исходного кода. \\
    \textbf{direction} теперь для большей ясности является \textbf{enum}. \\
    \textbf{\#ifdef DEBUG} внутри функций был изменен на макрос. \\
    Изменён алгоритм чтения и записи в соответствии с заданием. \\
    Код модулей приведён в приложении.

  \subsection{Сборка и подключение модуля}
    Создан \textbf{Makefile}:
    \lstinputlisting[language = bash]{driver/Makefile}

    Сборка модуля и подключение:
    \begin{lstlisting}
sudo make -C /lib/modules/`uname -r`/build M=$PWD modules
sudo insmod chardev.ko

##DMESG##
[45189.952399] SUCCESS: Registered the major device 235
[45189.952407] Please, create a device file to interact with the driver. Use:
[45189.952408] mknod chardev c 235 0
    \end{lstlisting}
    \newpage

  \subsection{Проверка корректности работы}
    Создан файл устройства для взаимодействия с драйвером:
    \begin{lstlisting}
sudo mknod chardev c 235 0
    \end{lstlisting}

    Проверка:
    \begin{lstlisting}
su 
cd /dev
echo "message message" > chardev
cat chardev
##RES## : message
echo "direction back" > chardev
cat chardev
##RES## : egassem
echo "direction back" > chardev
cat chardev
##RES## : message
echo "msg_delete" > chardev
cat chardev
##RES## :

##DMESG##
[45190.112732] Message has been written
[45190.112733] Read 5 bytes, 101207 left
[45190.112734] Forwards direction
[45190.112735] Read 5 bytes, 101207 left
[45190.112736] Backwards direction
[45190.112737] Read 5 bytes, 101207 left
[45190.112738] Message has been deleted
    \end{lstlisting}

\section{Выводы}
  В ходе работы был реализован и подключен драйвер.
  Взаимодействие с драйвером таким образом эффективно, поскольку можно взаимодействовать с ним, не изменяя код.
  \newpage

\section{Приложение}
  \subsection{Исходный код модуля chardev.h}
    \lstinputlisting[language = C]{driver/chardev.h}
  \subsection{Исходный код модуля initalize.c}
    \lstinputlisting[language = C]{driver/initialize.c}
  \subsection{Исходный код модуля open.c}
    \lstinputlisting[language = C]{driver/open.c}
  \subsection{Исходный код модуля release.c}
    \lstinputlisting[language = C]{driver/release.c}
  \subsection{Исходный код модуля read.c}
    \lstinputlisting[language = C]{driver/read.c}
  \subsection{Исходный код модуля write.c}
    \lstinputlisting[language = C]{driver/write.c}

\end{document}
