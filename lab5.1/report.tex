\documentclass[a4paper,11pt]{article}

\usepackage{wrapfig}
\usepackage{amsmath}
\usepackage{listings}
\usepackage{color}
\usepackage[dvipsnames]{xcolor}
%Russian-specific packages
%--------------------------------------
\usepackage[T2A]{fontenc}
\usepackage[utf8]{inputenc}
\usepackage[russian, english]{babel}
%--------------------------------------

\title{Отчёт по лабораторной работе №5.1 по дисциплине GNU/Linux}
\author{Андрей Бареков}
\date{\today}

\lstdefinestyle{mycode}{
    backgroundcolor=\color{White},   
    commentstyle=\color{ForestGreen},
    keywordstyle=\color{Blue},
    numberstyle=\tiny\color{Maroon},
    stringstyle=\color{Plum},
    basicstyle=\ttfamily\footnotesize,
    breakatwhitespace=false,         
    breaklines=true,                 
    captionpos=b,                    
    keepspaces=true,                 
    numbers=left,                    
    numbersep=5pt,                  
    showspaces=false,                
    showstringspaces=false,
    showtabs=false,                  
    frame=single,
    tabsize=2,
}

\begin{document}
\maketitle
\newpage

\section{Цель работы}
  Узнать размер символьной информации ядра при его сборке.

\section{Задачи}
  \begin{enumerate}
    \item Скачать стабильную версию ядра.
    \item Собрать ядро без использования символьной информации.
    \item Собрать ядро с использованием символьной информации.
    \item Вычислить разницу между размерами откомпилированных ядер.
  \end{enumerate}

\section{Платформы}
  \subsection{Аппаратная платформа}
    CPU: Intel i5-8250U (8) @ 3.400GHz

  \subsection{Программная платформа}
    OS: Linux Manjaro x86\_64
    Версия ядра: Linux 5.2.21

\newpage
\section{Ход работы}
  \subsection{Подготовка к сборке ядра}
    \lstset{style=mycode}

    Склонирована стабильная версия ядра 5.2.21 и скачаны необходимые исходники кода:
    \begin{lstlisting}
git clone https://gitlab.manjaro.org/packages/core/linux52
updpkgsums
    \end{lstlisting}

    Распакован архив с исходным кодом:
    \begin{lstlisting}
tar -xf linux-5.2.tar.xz
cd linux-5.2
    \end{lstlisting}

    Очищено дерево ядра:
    \begin{lstlisting}
make mrproper
    \end{lstlisting}

    Создан файл конфигурации для сборки ядра без символьной информации. Параметры оставлены по умолчанию:
    \begin{lstlisting}
make menuconfig
    \end{lstlisting}

    Измерено время сборки ядра, использованы 6 потоков:
    \begin{lstlisting}
time make -j6
...
make -j6  2780,43s user 306,34s system 490% cpu 10:29,24 total
    \end{lstlisting}

    Измерен объём собранного ядра:
    \begin{lstlisting}
du -m ./
...
1400	./
    \end{lstlisting}

    Размер - 1400 MB.
    
    Архив с ядром скопирован в другую папку, распакован и очищено дерево ядра:
    \begin{lstlisting}
mkdir withsyminfo
cp linux-5.2.tar.xz withsyminfo/
cd withsyminfo
tar -xf linux-5.2.tar.xz
cd linux-5.2
make mrproper
    \end{lstlisting}

    Создан файл конфигурации для сборки ядра с символьной информацией.
    Изменены параметры: в разделе настроек \textbf{Kernel hacking --> Compile-time checks and compiler options} включены параметры
    \textbf{Compile the kernel with debug info} и \textbf{Reduce debugging information}.
    \begin{lstlisting}
make menuconfig
    \end{lstlisting}

    Аналогично скомпилировано ядро и измерено время сборки при 6 потоках, измерен объём:
    \begin{lstlisting}
time make -j6
...
make -j6  3214,60s user 356,45s system 484% cpu 12:17,50 total
du -m ./
...
2308	./
    \end{lstlisting}

\section{Результаты измерений}
  \begin{enumerate}
    \item Время сборки ядра без символьной информации - \textbf{10m29s}.
    \item Объём ядра, собранного без символьной информации - \textbf{1400MB}.
    \item Время сборки ядра c символьной информацией - \textbf{12m17s}.
    \item Объём ядра, собранного c символьной информацией - \textbf{2308MB}.
    \item Разница мужду объёмами - \textbf{908MB}.
  \end{enumerate}

\section{Выводы}
  В ходе работы был измерен объем ядра, собранного без и с символьной информацией.
  Ожидаемо, время сборки и размер ядра получились больше при компиляции с символьной информацией.

\end{document}
