\documentclass[a4paper,11pt]{article}

\usepackage{wrapfig}
\usepackage{amsmath}
\usepackage{listings}
\usepackage{color}
\usepackage[dvipsnames]{xcolor}
%Russian-specific packages
%--------------------------------------
\usepackage[T2A]{fontenc}
\usepackage[utf8]{inputenc}
\usepackage[russian, english]{babel}
%--------------------------------------

\title{Отчёт по лабораторной работе №7 по дисциплине GNU/Linux}
\author{Андрей Бареков}
\date{\today}

\lstdefinestyle{mycode}{
    backgroundcolor=\color{White},   
    commentstyle=\color{ForestGreen},
    keywordstyle=\color{Blue},
    numberstyle=\tiny\color{Maroon},
    stringstyle=\color{Plum},
    basicstyle=\ttfamily\footnotesize,
    breakatwhitespace=false,         
    breaklines=true,                 
    captionpos=b,                    
    keepspaces=true,                 
    numbers=left,                    
    numbersep=5pt,                  
    showspaces=false,                
    showstringspaces=false,
    showtabs=false,                  
    tabsize=2
}

\lstdefinestyle{diffred}{
    backgroundcolor=\color{Salmon},   
    commentstyle=\color{ForestGreen},
    keywordstyle=\color{Blue},
    numberstyle=\tiny\color{Maroon},
    stringstyle=\color{Plum},
    basicstyle=\ttfamily\footnotesize,
    breakatwhitespace=false,         
    breaklines=true,                 
    captionpos=b,                    
    keepspaces=true,                 
    numbers=left,                    
    numbersep=5pt,                  
    showspaces=false,                
    showstringspaces=false,
    showtabs=false,                  
    tabsize=2
}

\lstdefinestyle{diffgreen}{
    backgroundcolor=\color{SpringGreen},   
    commentstyle=\color{ForestGreen},
    keywordstyle=\color{Blue},
    numberstyle=\tiny\color{Maroon},
    stringstyle=\color{Plum},
    basicstyle=\ttfamily\footnotesize,
    breakatwhitespace=false,         
    breaklines=true,                 
    captionpos=b,                    
    keepspaces=true,                 
    numbers=left,                    
    numbersep=5pt,                  
    showspaces=false,                
    showstringspaces=false,
    showtabs=false,                  
    tabsize=2
}

\begin{document}
\maketitle
\newpage

\section{Цель работы}
  Реализовать драйвер простого устройства в виде модуля ядра и подключить его на уровне ядра.

\section{Задачи}
  \begin{enumerate}
    \item Ознакомиться с представленным материалом.
    \item Исправить ошибки.
    \item Скомпилировать драйвер как модуль ядра.
    \item Подключить драйвер как модуль на уровне ядра.
    \item Продемонстрировать корректную работу драйвера.
  \end{enumerate}

\newpage
\section{Ход работы}
  \subsection{Исправление ошибок}

    Удалён макрос \textbf{PREPARE WORK}.
    У \textbf{INIT WORK} изменилась сигнатура, поэтому в эту функцию
    передается структура.

    \lstset{style=diffred}
    \begin{lstlisting}[language = C++]
    irqreturn_t irq_handler(int irq, void *dev_id, struct pt_regs *regs)
    {
      static int initialised = 0;
      static unsigned char scancode;
      static struct work_struct task;
      unsigned char status;

      status = inb(0x64);
      scancode = inb(0x60);

      if (initialised == 0) {
        INIT_WORK(&task, got_char, &scancode);
        initialised = 1;
      } else {
        PREPARE_WORK(&task, got_char, &scancode);
      }

      queue_work(my_workqueue, &task);

      return IRQ_HANDLED;
    }
    \end{lstlisting}

    \lstset{style=diffgreen}
    \begin{lstlisting}[language = C++]
    irqreturn_t irq_handler(int irq, void *dev_id, struct pt_regs *regs)
    {
      static struct work_struct task;
      unsigned char status;
      static unsigned char scancode;

      status = inb(0x64);
      scancode = inb(0x60);
      got_char(&scancode);

      INIT_WORK(&task, (work_func_t)got_char);
      queue_work(my_workqueue, &task);

      return IRQ_HANDLED;
    }
    \end{lstlisting}

    Изменилось название флага \textbf{SA SHIRQ} на \textbf{IRQF SHARED}.

    \lstset{style=diffred}
    \begin{lstlisting}[language = C++]
    return request_irq(1, irq_handler, SA_SHIRQ,
      "test_keyboard_irq_handler", (void *)(irq_handler));
    \end{lstlisting}

    \lstset{style=diffgreen}
    \begin{lstlisting}[language = C++]
    return request_irq(1, irq_handler, IRQF_SHARED,
      "test_keyboard_irq_handler", (void *)(irq_handler));
    \end{lstlisting}

  \lstset{style=mycode}
  \subsection{Компиляция}
    В папке с исходным кодом модуля создается \textbf{Makefile} для
    сборки модуля:

    \begin{lstlisting}
    obj-m += interrupt-handler.o
    \end{lstlisting}

    Компилируется драйвер как модуль ядра, приведен пример для Arch-подобных систем:

    \begin{lstlisting}[language = bash]
    sudo make -C /lib/modules/`uname -r`/build M=$PWD modules
    \end{lstlisting}

  \subsection{Подключение драйвера как модуля на уровне ядра}

    Драйвер подключается командой:
    \begin{lstlisting}[language = bash]
    sudo insmod interrupt-handler.ko
    \end{lstlisting}

    Чтобы не повредить файловую систему, открывается новый эмулятор терминала
    и вводится команда, которая перезагрузит систему через определенный отрезок времени:
    \begin{lstlisting}[language = bash]
    shutdown -r 120
    \end{lstlisting}

  \subsection{Проверка корректности работы модуля}

    После нажатия клавиш, в логи ядра записываются их скан коды.
    Открывается файл логгирования ядра:
    \begin{lstlisting}[language = bash]
    sudo dmesg
    \end{lstlisting}

    И показывается корректность работы драйвера:
    \begin{lstlisting}
    Scan Code 34 Pressed.
    Scan Code 34 Released.
    Scan Code 56 Pressed.
    Scan Code 56 Released.
    Scan Code 71 Pressed.
    Scan Code 71 Released.
    \end{lstlisting}

\section{Выводы}
  Был реализован и подключен драйвер в виде модуля на уровне ядра.
  В процессе достижения данной цели, были исправлены ошибки в коде
  обработчика прерываний, скомпилирован, подключен драйвер и показана
  его корректная работа.

\end{document}
